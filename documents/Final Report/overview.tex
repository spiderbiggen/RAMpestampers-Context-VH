\chapter{Overview of the software product}\label{ch:overview}
In this chapter we will give a brief overview of the product design and implementation. We will briefly outline how it works and some of its limitations.
\\
\\
The developed product consists of two parts, a runtime environment and a GOAL project.
The environment is used to connect with the Tygron engine and handles everything we need to make our GOAL project talk with the Tygron engine.
It takes care of data transmission between the agent and the Tygron engine.
The environment is built on the Tygron sdk, for communication with the engine, and the tygronenv\footnote{\url{https://github.com/eishub/tygron}} which is extended to provide additional functionality.
The functionality of the environment is limited by the actions and data available in the sdk.
Changes to the connector extend the default implementations and add some of our own, like new percepts, lists of objects for use in GOAL, and custom actions.
The changes to these percepts include adding more information to buildings and requests.
The custom actions are designed to perform a complex action that would be (almost) impossible in goal and then add the result as a percept so that it's result can be used in GOAL.

The GOAL project does all the decision making, it runs in a loop until all its goals have been achieved or its manually shutdown.
It is separated in modules so every distinct function has its own module, i.e. constructing or demolishing are placed in separate modules.
Every cycle the agent will call the event module first which will update the stored data with data it gets from the environment, for example a list of building.
It has one main module which decides what module should be called each cycle, based on its current goals.
Sometimes a module is called that cannot do anything because there is no space to build a construction, it will then adopt a goal to make some space so that it can eventually do something in the module that could not do anything.
The goal project runs mainly on SWI-Prolog\footnote{\url{http://www.swi-prolog.org/}} which is why it cannot easily handle complex arithmetic operations. 
This is why these need to be handled in the environment as mentioned before.

