\chapter{Overview of the software product}\label{sec:overview}
In this chapter we will give a brief overview of the product design and implementation. We will briefly outline how it works and some of its limitations.
\\
\\
The developed product consists of two parts, a runtime environment and a GOAL project.
The environment is used to connect with the Tygron engine and handles everything we need to make our GOAL project talk with the Tygron engine.
It handles translating the xml objects, retrieved through Tygron's backend, to parameters we can use with GOAL and vice versa.
It also handles the sendings of actions to the Tygron engine, like buying ground or constructing a building, this can be as simple as translating the parameters, but in certain cases this also handles some of the more complex arithmetic operations like retrieving a list of Multipolygons, a collection of points that form a 2D shape, that can be used to construct buildings.
The environment is built on the Tygron sdk, for communication with the engine, and Eishub eis\footnote{\url{https://github.com/eishub/eis}}, for the communication with GOAL, and is derived from the tygronenv\footnote{\url{https://github.com/eishub/tygron}}.
The functionality of the environment is limited by the actions and data available in the sdk.

The GOAL project does all the decision making, it runs in a loop until all its goals have been achieved or its manually shutdown.
It is separated in modules so every distinct function has its own module, i.e. constructing or demolishing are placed in separate modules.
Every cycle the agent will call the event module first which will update the stored data with data it gets from the environment, for example a list of building.
It has one main module which decides what module should be called each cycle, based on its current goals.
Sometimes a module is called that cannot do anything because there is no space to build a construction, it will then adopt a goal to make some space so that it can do something in that module.
The goal project runs mainly on swi-prolog which is why it cannot easily handle complex arithmetic operations, this is why we need to handle these in the environment as mentioned before.
