\section{Overview of the software product}
The developed product consists of two parts, a runtime environment and a goal project.
The environment is used to connect with the Tygron engine and handles everything we need to make our goal project talk with the Tygron engine.
It handles translating the xml objects, retrieved through Tygron's backend, to parameters we can use with goal and vice versa.
It also handles sending the actions to the Tygron engine, like buying ground or constructing a building, this can be as simple as translating the parameters, but in certain cases this also handles some of the more complex arithmetic operations like retrieving a list of Multipolygons, a collection of points that form a 2D shape, that can be used to construct buildings.

The GOAL project does all the decision making, it runs in a loop until all it's goals have been achieved or it's manually shutdown.
It is separated in modules so every distinct function has it's own module, i.e. constructing or demolishing are placed in separate modules.
Every cycle it will call the event module first which will update the stored data with data it gets from the environment, like a list of building.
It has one main module which decides what module should be called each cycle, based on it's current goals.
Sometimes a module is called that can't do anything because there is no space, it will then adopt a goal to make some space so that it can do something in that module.