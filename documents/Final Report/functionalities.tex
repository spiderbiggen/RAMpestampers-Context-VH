\section{Description of the developed functionalities}
In this section the functionalities of our agent will be described. The functionalities are based on the strategy of our agent. This strategy consists of two parts, each part will be explained in the following subsections.

\subsection{Building strategy}
In this part the strategy about the decision of building constructions and which constructions to be build will be explained. In the first section the reason to build a construction will be defined, and in the second section the choice for a particular construction will be clarified. 

\subsubsection{Reasons to build a construction}
The first reason to build a construction, is that it can influence an indicator in a positive way. Indicators define your score in the game and a high indicator shows that you are doing well. Another reason is that a building can give you profit in terms of your budget.

\subsubsection{Influence of building a construction}
There is one direct reason to build a building. Our vision as TU Delft is to create a modern campus with luxery high buildings, so we created an indicator that increases when we build these buildings. The more high buildings, the better the indicator will be. Another indicator is about extending the education space. To achieve this we need to build more education buildings. Another less important indicator is the amount of green around the campus. When there is no room for a building, we build trees to get a better score with the green indicator. The agent will build constructions as long as the budget holds and as long as it gives a positive influence on the indicators.

\subsubsection{Which construction to build}
Because a part of our vision is to build high buildings, the buildings that our agent builds, needs to be high buildings. In the Tygron engine, high buildings are of the type “Luxe variant”, so these buildings needs to be built until the indicator has reached 100\%. When the indicator has reached 100\%, we can also build other education buildings. For the extension of the education space, it does not matter which building is built, but we try to build buildings that are the most beneficial for the indicator.
For the green indicator there are a lot of different trees and parks, and they all have the same effect on the indicator, so it does not matter which one the agent builds. We decided to only build 'loofbomen', because they look nice and are realistic for the TU Delft to build. So, there is not an advanced tactic to do this.

\subsection{Demolish strategy}
In this section the strategy about destroying buildings will be explained. In the first section the reason to destroy a building will be defined, and in the second section the choice for a particular building will be clarified. 

\subsubsection{Reasons to demolish a building}
There are two reasons to demolish a construction and both of them ultimately give a better indicator score. The first reason is to directly influence an indicator, and the second reason is to make space for a new building that influences an indicator. Both will be discussed in the following sections. The agent needs to keep destroying buildings in order to build newer, higher buildings, until all indicators are at 100\%.

\subsubsection{Direct influence}
There is one direct reason to destroy a building. We created an indicator 'Destroy old buildings' that has the goal to destroy all pre-war type of buildings, to create a modern campus. There are quite a lot of these buildings, and if they are all destroyed this indicator will get to the 100\%. This is a reason to just destroy some buildings without needing another reason and it has a direct influence on an indicator. However, just destroying buildings also has a downside: it will possibly lower the other three indicators. These are to have a positive budget, build more education buildings and build high education buildings. So: if you destroy a certain building there is a chance it will impact the total score negatively. Our agent should try to find a balance here and try to keep our overal score as high as possible.

\subsubsection{Indirect influence}
Sometimes when a building is destroyed, there is no direct benefit. But the created space is needed to built a new building that improves the score of the building indicators. This is explained in the building strategy section, but it is important to realize when destroying something that there will be new land to build something on. So even when destroying something is not immediately beneficial, there is a possibility it will be beneficial because there is a new area to build a building on.

\subsubsection{Which building to destroy}
There is only one indicator that benefits from destroying buildings, and that indicator is the 'Destroy old buildings' indicator. So when a building needs to be destroyed for a reason in the previous section, an old building should be chosen. Those buildings have the pre-war type. Our agent should check which of the buildings are the most beneficial to destroy, some pieces of land are more usefull for building then others. Currently our agent does not look at this and assumes he has to destroy all buildings anyway. 

\subsection{Sell land strategy}
In this section the strategy of selling land will be explained. 

\subsubsection{When to sell land}
If the agent destroys a building and finds out that the piece of land that he is left with is not suitable for buildings, it will try to sell that piece of land to other stakeholders. Since the environment can not handle multiple sell requests to different stakeholders simultaneously, we will try to sell it to one stakeholder at a time. After a few cycles we will assume the agent did not respond or declined the offer and we will try to sell the land to another stakeholder. If we've tried selling it to every stakeholder we will lower the price and start over. This way we will try to sell it for the maximum value, increasing our budget.

\subsection{Request strategy}
This section describes the strategy behind the requests.
\\\\
The requests are either accepted or declined. A request is declined when another agent wants to buy a piece of land that is planned for building a construction. Other requests that are beneficial for the agent’s indicators are accepted.

