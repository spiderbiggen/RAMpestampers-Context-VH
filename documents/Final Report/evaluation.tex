\section{Evaluation of the functional modules and the product in its entirety}
In this chapter an evaluation is given of the product in its entirely, including a failure analysis. 

\subsection{Evaluation on the product}
The product in its entirely as we have now, is not exactly as we had planned at the beginning of the project. There are some positive and negative outcomes of what we had planned to achieve. The negative outcome is especially that we did not have the time for implementing all the decision logic in the agent that we wanted. The reason for this will be explained in the failure analysis section. The positive outcomes of the product is that we have implemented some features that was not planned. Selling land is a feature that we implemented, but was not planned beforehand. With selling land we earn money with land we do not use.  

\subsection{Failure analysis}
The connector is the main reason why we have not as much decision logic as we wanted. We had some misunderstanding about the connector at the beginning of the project. When starting this project, our thoughts were that the connector was almost fully functional to implement a good agent. When making the planning and vision about our agent, we had in mind that we could start in the first weeks of the project with implementing the agent.  This was not the case, the connector was missing some functionalities. The most important feature that was missing, were some important percepts. \\\\
Besides the implementation part of the connector, the communication between groups have caused some delays too. In the first few weeks, we had only one meeting with all the groups together. The rest of the time, we worked with our own group together and communication between groups was done without face to face communication. This resulted in a very late response time for small issues, which caused some process loss. After sprint 5, we had two meetings every week and made some clear arrangements about responding to pull requests and issues about the connector.
