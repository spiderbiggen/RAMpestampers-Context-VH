\chapter{Outlook}
In this section we will talk about the future of the product and what are enhancements for the connector and goal code that could improve the product. Also we will talk about the virtual humans context project and what we think could improve the project for future students.
\\
\\
\section {Product}
Our current agent has no way to find out how a certain indicator can be improved, we had to hardcode the names of the indicators. This could be a big improvement to make the agent a lot more dynamic. Changes in the connector and possibly in the SDK are needed for this, but it would make programming an agent a lot more convenient.

Another improvement would be to change the way land works, our agent has to work with multipolygons currently which makes most computations impossible. Because of this a lot of functionality is moved to the connector. Having to make a function call to the environment for every calculation with land does not work very well for an agent that has to do city planning. If we get a request to buy a piece of land for example, we have to know if there are buildings on this land, if it has a good shape to build buildings and where the land is exactly, to make a good decision about this. But all of this information our agent can not figure out by itself. A grid of the landscape where every building is build within cells would be a solution for this, making it a lot easier to see collisions and locations. This would require a lot of changes to the connector though, but would enable our agent with a lot more possibilities.

To get our agent to behave more like a human there are some enhancements we could make as well. Our current agent has a very specific strategy which will execute in exactly the same order most of the time. This is something that could be improved, as this makes the agent predictable, where a human would act more randomly. The agent has also very fast decision making, where a human would take its time to respond, the agent responds immediatly to most things. When playing as a human this seems a bit odd, and a delay should be introduced.
\\
\\
\section {Project}
We did have fun during this project, but there were also some problems which caused a lot of frustration. The biggest issue was probably that we did not have a good understanding about the functionalities of the connector. We expected it to have most basic functionalities, and that we could later implement some more advanced functionalities if needed. But it turned out that we still had to implement most of the functionality. A good improvement for next year would be to start working on the connector in the first week. It would also be very usefull to have a session in which the connector would be explained. Although working with such a big group has its problems, we do think this should still be done in the next years, it was a big learning experience and we learned a lot about communication and working on a software product with a lot of people. An improvement would be to have a weekly meeting with all the groups and the TAs. We did start having these at the end of the project, but these were a bit chaotic sometimes and not every group made good agreements during this meetings. Another improvement would be to have a stable goal version in which tests and the debugger would work consistently, this caused a lot of delays this year. The last few weeks none of the groups were able to use the debugger succesfully in eclipse and this made fixing bugs a lot harder then it has to be.