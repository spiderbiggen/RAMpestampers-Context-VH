\section{Product Attributes}
Like stated before, the goal of this product is to replace people in the Tygron engine using agents. These agents should be able behave like a human being. There are several attibutes in our agent that will help to make this agent perform like a normal human being. Our agent needs to pursue a goal, by performing actions on the map and negotiating with other agents.

\subsection{Goal}
The goal of each entity in the game is set beforehand. Our agent will represent the TU Delft in the TU-wijk. Here the TU Delft has several goals and interests. For example there are a few faculties that need to be rebuilt or renovated and this needs to be done within the budget available. A couple of these faculties are EWI\cite{tunietpuz} and CiTG\cite{turen}. Also, the TU Delft wants to build a new faculty and keep the place as green is possible. All these goals are normally tried to accomplish by humans, but this time our agent will try to accomplish them. He needs to do this by working like a human would. So the goals are the first attributes that are needed to satisfy the customer.

\subsection{Actions}
There are four main actions in the game. These actions represent human actions to get things done in the world of Tygron. The actions are \textbf{build}, \textbf{destroy}, \textbf{buy} and \textbf{give money}. All the actions are needed to accomplish specific goals, so the implementation of the agents will mainly consist of reacting to the environment by doing one of the previous actions. This means the actions are really important for the game and normal humans would also use these options all the time. This means optimizing when to use them is very benificial and that makes it an important attribute.

\subsection{Communication}
Not all actions can be done without permission from another entity in the game. This entity can be another person or an agent. So there needs to be some sort of communication to get permission from other agents. Without this, it's impossible to achieve your goals. For example, if the TU Delft want to build a new faculty on the ground of the town, it needs permission to do so. And the major might have certain demands on the building, for example a limit on the amount of stores. All these things need communication to happen, so it is important to have communication in the game as one of the elements. Actual humans playing the game communicate a lot by directly talking to eachother to make deals. An agent obviously can't talk, but can also send messages in game. So the last attribute to simulate humans using agents is to let the agents communicate with eachother. A special language could be written to do so: so there needs to be an ontology\cite{ontolagents}.
