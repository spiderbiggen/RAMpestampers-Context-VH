%Beschrijving van een bruikbaarheidsevaluatie uitgevoerd met het product 
%we gaan hiervoor een turing test uitvoeren, deze zullen we in deze sectie beschrijven

\section{Evaluation}
In this section the interaction of a user will be evaluated according to various techniques as described in the Interaction Design course. This will be done by using both a cognitive walkthrough and an emprical method in the form of an expiriment.

\subsection{Cognitive Walkthrough}

In the earlier parts of this project the method cognitive walktrough was used to deduce multiple strategies for the agent. After all the agent needs to be able to make logical choices and be able to respond to the other stakeholders (played by one or more users) in a logical way. This means that there are multiple scenarios in which a user communicates with the agent that should have a clear sequence of actions as a result. For instance, when the user asks to buy land fromt the agent while it is still planning to build on that land or if the price offered is not acceptable will result in the following action sequence: 
\begin{enumerate}
\item Select a piece of land to buy from the agent
\item Choose a price to offer for the land
\item Place request to buy availible land from the agent
\item Wait for a response
\item Click the popup saying that the agent declines the offer
\end{enumerate}

This sequence is one of many to which the agent will be able to formulate a logical answer. In the ideal case the agent will give responses that correspond 100\% to how the person that the agent is replacing would give. In the next part the actual interaction between a human and the agent will be assessed using a Turingtest.

\subsection{Turingtest}
This part of the evaluation will explain about the experiment that we have conducted to test the interaction of our agent with a human player. It contains the method that we used to assess the interaction, the results that we have found and the conclusion that we can draw from our results. 

\subsubsection{Method}
The experiment that was put together was quite simple to do. It consists of a test person behind a computer and one of the people conducting the experiment behind a computer as well. The participant joins a session in the Tygron environment as the municipality stakeholder while the experimenter joins the session as the TU Delft stakeholder. The participant does not know whether the experimenter had joined himself or joined with the agent. The participant then has to do a few simple tasks such as trying to buy ground from the TU Delft, sell land or respond to any of the requests that the TU Delft makes. The participant was asked to think aloud while making decisions and reacting to the actions of either the person or the agent. At the end of the run the participant will tell if they think the TU Delft was played by a human or an agent and how confident they are on a scale of 0 to 100. This process was repeated multiple times. \\

The test persons that were chosen for this experiment were the parents of the experimenters because our product (the agent) would be used to replace a certain stakeholder and playing together mostly with middle-aged people. On top of that, both of these groups will most likely have zero to no experience with playing together with an agent in the Tygron engine. The expiriment was conducted on 3 different people for 5 separate rounds. The order of choosing between joining as human or agent was the same for every participant: human, agent, human, agent, agent.

\subsubsection{Results}
These are the results that were found: \\

\textbf{Person 1}
\begin{tabular}{lll}
\textbf{Round} & \textbf{Choice}  & \textbf{Confidence}\\ 
1 & agent  & 20         \\
2 & human  & 15         \\
3 & human  & 40         \\
4 & human  & 40         \\
5 & agent  & 70        \\\\
\end{tabular}

\textbf{Person 2}
\begin{tabular}{lll}
\textbf{Round} & \textbf{Choice}  & \textbf{Confidence}\\ 
1 & agent  & 10        \\
2 & human  & 30         \\
3 & agent  & 80         \\
4 & agent  & 100         \\
5 & agent  & 100       \\\\
\end{tabular}

\textbf{Person 3}
\begin{tabular}{lll}
\textbf{Round} & \textbf{Choice}  & \textbf{Confidence}\\ 
1 & human  & 18         \\
2 & human  & 25         \\
3 & agent  & 25         \\
4 & human  & 40         \\
5 & agent  & 50        \\\\
\end{tabular}

As stated in the method section, we asked the participants to think aloud while making decisions. During the experiments we found that the participants had trouble early on to distinguish between the human and the agent, mostly because they were new to the Tygron engine. They were given assistance and tips on how to use the environment and how to do certain actions. As soon as they started to understand the environment better and recognize the things that were done by the other stakeholder that they began to get more confident about guessing whether they were playing with a human or an agent.

\subsubsection{Conclusion}
The results of this experiment were mostly corresponding to the results that were expected. For a person that has little to no experience with using the Tygron engine, the actions of another stakeholder may not be sufficient to tell if the stakeholder is played by a human or an agent. This is why in the first few rounds of the experiment the participants did not get the right answer more often and had low confidence in their choice. \\ The agent has a clear strategy that it will almost always execute in the same order. Because of this after a few iterations of the experiment some of the participants started to see a pattern in the actions that were executed. This led to the participant being able to easily distinguish between the humand and the agent after some time and also a higher confidence score.


