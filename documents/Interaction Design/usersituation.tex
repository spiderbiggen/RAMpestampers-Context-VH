\section{User}
This section will describe our user and how he/she uses the system. 
It starts with a description of the persona. 
After that a few user scenario's will be described.
Finally the tasks will be analysed.

\subsection{Persona}
This section describes the user by giving a description of the persona.
\subsubsection*{Hank - Student housing}
Hank is someone who is in the planning phase of a construction project, for a student housing company, around the TU Delft campus. He wants to use the Tygron engine for detailed information on possible outcomes of scenarios that can happen during his project, because the Tygron engine can provide him with some insights for this decision before he actually has to spend any money on the actual project. 

He can join a session and play the role of student housing company. The project will be a virtual version of the area surrounding the TU Delft. In this virtual version a lot of settings have been configured to match the reality. He has a budget and can do certain actions. He can also see indicators that can help him visualize his current goals and the state of these goals, they also provide some sort of score system. This score system will be shown at the end of the planning session and shows you how much of your own goals you completed. As housing company he wants to build more housing for students for example, but he has to come to an agreement with the other stakeholders before he is allowed to do some actions.

\subsubsection*{Lisa – Municipality}
Lisa works for the municipality of Delft. She wants to make sure that Hank can complete his project, but he can’t break any laws with his project otherwise she has to say no. Since Hank and Lisa need to work together, Lisa can also join the virtual environment of Tygron. She also has some indicators that define her score, like adding green around the campus and not allowing buildings of certain height. As municipality she has an important role becuase she needs to give permissions when buildings get build or destroyed for example.

\newpage
\subsection{Usage scenario}
The Tygron engine is designed to use in scenarios where all parties involved with the project for instance: the user (project lead), municipalities, housing cooperatives, etc. Each party makes decisions about actions that involve their party, for instance municipalities need to decide if you are allowed to construct a building, if the building is to tall, if you can buy a piece of land for your project, etc. But what if you decided that you wanted to try something in the Tygron engine, but not all other parties are available, you then have to either find someone who can fill in for the other parties or you postpone to a later date.

Our project can help with this by replacing a real world party with a computer controlled party. This computer controlled party, from now on referred to as agent, is designed to follow the guidelines that it is given through the engine, but it can also negotiate with other parties, be they human or computer controlled. The actions these agents take aren’t necessarily 100 percent accurate to what the party would actually do, but we try to make it as close as possible.

\newpage
\subsection{Task Analysis}
The main problem we want to work on is to make it possible for users to use the Tygron engine while not all other parties are present or ultimately when a user is alone. We want to achieve this by creating agents that will act like the party they replace. We will connect to the Tygron engine by using their SDK. 

We are to use Goal as our platform for creating these agents. Goal is designed for creating (somewhat) Artificial intelligence. It has easy goal management and also has a clear system that acts like the memory of the agent. This memory, also called the believe base, can also store any information the agent receives. 

One of the limitations of goal is how it interacts with other software. We were supplied with a “connector” for this that will handle the communication between goal and the Tygron SDK. Another limitation is its ability to handle mathematical problems. Therefore it can be difficult for the agent to figure out where to construct a new building. This Limitation can be overcome by adding a new action to the “connector” which specifies where it can build.
