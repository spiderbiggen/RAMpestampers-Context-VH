\documentclass{article}

\usepackage{graphicx}

\author{Job Zoon}
\title{Strategy when to destroy buildings}

\begin{document}
\pagenumbering{gobble}

\maketitle{}
\newpage{}

\section{Introduction}
In this document the strategy about destroying buildings will be explained. In the first section the reason to destroy a building will be defined, and in the second section the choice for a particular building will be clarified. This document will be used to implement the following user story:\\
\\
\textit{"As an agent\\
When in a simulation\\
I am able to destroy stuff if I think 
this is benificial for me\\
To make room for construction or to 
optimize a certain indicator."\\}

There are two important parts in this user story that need to be discussed before implementing anything. The 'stuff' part: so which buildings need to be destroyed, and the 'benificial' part: so when buildings need to be destroyed. Again, these two questions will be discussed in the following two sections.

\section{When/why to destroy buildings}
In this section the reasons to destroy a building will be explained. There are two types of reasons and both of them ultimately give a better indicator score. The first reason is to directly influence an indicator, and the second reason is to make space for a new building that influences an indicator. Both will be discussed in the following sections.

\subsection{Direct influence}
There is one direct reason to destroy a building. We created an indicator 'Destroy old buildings' that has the goal to destroy all pre-war type of buildings. There are quite a lot of these buildings, and if they are all destroyed this indicator will get to the 100\%. This is a reason to just destroy some buildings without needing another reason and it has a direct influence on an indicator. However, just destroying buildings had also a downside: it will possibly lower the other three indicators. These are to have a positive budget, build more education buildings and build high education buildings. So: if you destroy a certain building there is a chance it will impact to total score negatively. The only way to make it possitive is to build a building on that ground: see the following section.

\subsection{Indirect influence}
Usually when a building is destroyed, there is no direct benefit. That's why there needs to be built a new building on that spot that improves the score of the building indicators. This will be further explained in the building strategy document, but it is important to realize when destroying something that there will be new land to build something on. So: even when destroying something is not immediately beneficial, there is a possibility it will be beneficial because there is a new area to build a building on.

\section{Which building to destroy}
This section is relatively easy compared to the previous one. In this section it will be explained which building should be destroyed. There is only one indicator that benefits from destroying buildings, and that indicator is the 'Destroy old buildings' indicator. So: when a building needs to be destroyed for a reason in the previous section: an old building should be chosen. Those buildings have the pre-war type. It doesn't really matter which of the old buildings to destroy. 

Then, if all old buildings are already destroyed and there still needs to be destroyed more, there is one more reason to chose a specific building. There is one indicator that needs high buildigs to get a higher score. So: if all old buildings are destroyed the next building to destroy is the lowest one, to make room for a high one.

\section{Summary}

\end{document}